---
title: Before we begin ...
author: Dimitri Merejkowsky
date: EFREI 2022-2023
---

# Knowing each other

## Me

Dimitri Merejkowsky

\vfill

Software developer for more than 15 years

\vfill

Feel free to write to me about anything:

dimitri.merejkowsky@intervenants.efrei.net

## My goals

* Prepare you for the projects to come.

* Teach you what I learned from the industry.

## Your turn

Please fill out the following survey:

https://forms.office.com/r/6jddy8gCNs

# Opinions and Facts

Everything a teacher tells you is a combination of opinions and facts.

## Facts

I'll try and tell you things that are *true* (or at least that I think they are true)

But I can make mistakes!

## Opinion

I also have a personal view on certain topics.

I'll try to make it clear when I'm not being factual, but I may forget.

Beware!

# Rules

## On being late

Don't be late with a good excuse!

It's not tolerated well in the industry, and
it can get you a bad reputation

You'll miss cool stuff.

## Tell me when something is wrong

If anything is wrong, you have to *tell me*.

Some examples:

* I'm going to fast
* The code samples I'm showing don't make sense
* You're stuck during a workshop
* You have a question
* ...

## On plagiarism

You are not allowed to plagiarize code. That is, pretend
you wrote code that is not yours.

But, you can (and should) use code from the Internet or
written by your class mates - just say so and include
a link to the source when necessary.

## My duties

* Do my best so that we achieve our common goals
* Be here to help you


## Your rights

You're allowed to make mistakes, it's often a very
good way to learn.

I should adapt my course to you, not
the other way around!

## Your rights (2)

You can interrupt me any time for any reason : if I'm too slow, if I'm too fast,
if you have a question ...

I prefer when it's interactive, and it will benefit your classmates.

I may save your question for a later time, though - don't take offense at that :)

# Schedule & Evaluations

## Schedule

* Two lectures - 2 sessions - 6 hours total
* Workshops - 3 sessions - 12 hours total
* Project - 3 sessions - 10 hours total

## Evaluations

You'll get 3 grades:

* 30% - Workshop Quiz
* 30% - Project
* 40% - Final exam

## From workshops

You'll get a 20-minutes quiz during the last workshop,
about what you learned during the workshops

I'll ask you to send me the code you write during
the workshops, but it will *not* be used for grading.

## Project - teams

* Teams of 4 members minimum
* All team members get the same grade

## Project - sessions

You'll have to:

* write test code
* write a *log* describing what you did and why
* produce various deliveries

Deliveries to be sent by mail at the end of each session.

## Project - evaluation

Taken into account:

* Test code:
    * quality
    * coverage

* Other deliveries:
  * readability
  * how well you put things into perspective

## Final exam

* Written exam
* 2 hours
* Some questions about what you learned during lectures and workshops
* An essay : you'll have to show you are capable of thinking about tests

## Your evaluation of this course

Very important ! This will help the course on Software Testing get
better - and this will help your fellow students :)

# Finally

This course is open-source :)

Find the sources of this slides (and more) here:

https://github.com/dmerejkowsky/2022-2023-efrei-st2tst
