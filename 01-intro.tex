---
title: Part 1 - Introduction to testing
author: Dimitri Merejkowsky
date: EFREI 2022-2023
---


# What is at stake

## An old technique

Testing has been part of the industry forever!

\vfill

Let's start with an example.

## Testing airplanes during storms

During a storm, an airplane will experience two things:

* Massive intensity
* Huge voltage

\vfill

We know how to produce massive intensity and huge
voltage in a lab, but not at the same time. What to do?

##

![](img/airplane-1.png)

##

![](img/airplane-2.png)

##

![](img/airplane-in-storm.png)

##

* First, we build a model and check where the electric current goes
* Then we take a part of the cabin and check what happens when the current goes through
* If those two tests work, we send a test pilot in the storm

## Software is eating the world



![](img/eating-the-world.jpg)

## Gravity of bugs

 * Ariane 5
 * Unintended acceleration
 * X-Ray machine - Therac 25

##

After quite some time, we now know that *humans are not capable* of
writing bug-free code.

## Java

```java
public String process(int input) {
  if (value % 3 == 0) {
    return "Hello";
  }
  // Bug! Missing `else`
}

String value = process(5);
String result = value.toUpperCase();
```


\vfill
Error at **compile** time:


```text
missing return statement
```

## JavaScript

```javascript
function process(value) {
  if (value % 3 === 0) {
    return "hello";
  }
  // Bug ! Missing `else`
}

const value = process(5);
const result = value.toUpperCase();
```

\vfill
Error at **runtime**:

```text
Uncaught TypeError: Cannot read properties of
undefined (reading 'toUpperCase')
```

## Productivity / reliability

![](img/productivity-reliability.png)

##

The compiler alone won't save you.

## Demo

## FizzBuzz

Rules:

Count from 1 to 100

* If divisible by 3, print "Fizz"
* If divisible by 5, print "Buzz"
* If divisible by 15, print "FizzBuzz"
* Otherwise, print the number

# Requirements

## Introduction

> We need you to draw seven red lines, all of them strictly
> perpendicular, some with green ink, and some with transparent.


The expert

=> https://www.youtube.com/watch?v=BKorP55Aqvg


## Definition

Requirements are statements specifying exactly what it
is that a piece of software should do under certain
conditions.

Requirements ensure that the developers know what to
build, and the testers know what to test.

Note that there are other methods of determining what
software to build aside from traditional requirements,
such as user stories.

## Definition of software testing through requirements

Software testing is the process of applying *metrics* to
determine product *quality*.

It is the *execution* of software and the comparison of the
results of that execution against a set of pre-determined
criteria.


## Requirements and domain

Depending on domain, you may want stricter requirements.

Nuclear power-plant software is not the same as a mobile game.

## Testability - 1

The bad requirements state how something should be
done, not what the system needs to do.

From a tester’s perspective, one of the most important
aspects of requirements is whether they are testable.

From a software development life cycle perspective,
“testable requirements” is another term for “good
requirements”. **A requirement that cannot be tested
cannot be shown to have been met**.

## Testability - 2

For the requirements to be testable, they must meet five
criteria, each of which we will cover individually. They
should be:

1. Complete
2. Consistent
3. Unambiguous
4. Quantitative
5. Feasible to test

## Consistency

The requirements should not contradict each other or the laws of the
product's domain.

## Unambiguous

Requirements should specify things as *precisely* as possible for working
in the particular domain the software is for.

Note they'll always be room for a *little* ambiguity

## Quantitative

The requirements should be **quantitative** (as opposed to
**qualitative**).

If you can apply numbers to a requirement, you should.
You should avoid using any sort of subjective terms
such as “fast”, “responsive”, “usable”, or “beautiful” .


## functional vs non-functional requirements

* Functional : what a system should do

* Non-functional: What a system should be

## Example of non-functional requirements

> The system shall be usable by an experienced computer user
with less than three hours’ training.

\vfill

> The system shall be able to support one hundred simultaneous
users.

\vfill

> The system shall be reliable, having less than one hour of
unanticipated downtime per month.

## Testability

Functional requirements are (relatively) simple to test
because  they say that a specific behavior should occur under
certain circumstances.

Non-functional requirements are harder to test because
the conditions are vaguer. But you can try and
*quantify* non-functional requirements

## Implicit requirements

Don't forget those!

For instance, people expect there data to stay **private** and **safe**,
but the is often missing from the requirements.

You can also come upon your own requirements, based on your
experience and what you know about the domain

## Specification

A set of requirements in a written form.

## Writing specifications

A.k.a: Requirements engineering

It takes **a lot of text** to remove ambiguity.

Changes are often difficult to make

## Specifications in real life - 1

![](img/requirements-01.jpg)

## Specifications in real life - 2

![](img/requirements-02.jpg)

## Specifications in real life - 3

![](img/requirements-03.jpg)

## Specifications in real life - 4

The spec you'll deal with  will often be incomplete and likely contains
errors.

\vfill

The correct way to deal with it is to *refine* the requirements with
the authors of the spec


# Defects

![](img/major-bug.png)

## Definition

A defect is any sort of error in a
program which causes the system
under test to do one of the
following:

1. Not meet the specified
requirements (functional or
non-functional)

2. Return an incorrect result

3. Stop execution unexpectedly
(system stability is an implicit
requirement in all systems
under test)

## Happy path / sad path

Aka success case, failure case

Success case:  returns an expected result given an input

Failure path : check what happens when the input is incorrect


# Testing Approaches


## Black -, White-, and Grey-box testing

![](img/black-gray-white-testing.png)

## Black-box

The tester has no knowledge of the internal workings of the system and
accesses the system as a user would. In other words, the tester does not
know about what database is in use, what classes exist, or even what
language the program is written in. Instead, testing occurs as if the
tester were an ordinary user of the software.

## White box

The tester has
intimate knowledge of the codebase and directly tests
the code itself. White-box tests can test individual
functions of the code, often looking at much more
granular aspects of the system than black-box tests.

## Gray box testing

Grey-box testing is a hybrid approach between white
and black-box testing. Grey-box testing involves
accessing the system as a user (as a black-box tester
would do), but with knowledge of the codebase and
system (as a white-box tester would have). Using this
knowledge, the gray-box tester can write more focused
black-box tests.

# Testing levels

## Smoke testing

A smoke test involves a minimal amount of testing that
can be done to ensure that the system under test is ready
for further testing.

It can be thought of as a guard to further testing—unless
the system can perform some minimal operations, you
don’t move on to running a full test suite.

This is usually a small number of tests, checking that the
software can be installed, that major functionality seems
to work, and that no obvious problems appear.

Smoke testing allows you to determine whether the
software is even worth testing before running the full
test suite on it.

## Smoke testing - example

For example, if you are testing an e-commerce site, you
may check to see that a user can access the site, search
for an item, and add it to their shopping cart.

The full test suite would check numerous edge cases and
paths, as well as all of the ancillary functionality such as
recommendations and reviews.

## Acceptance testing

Acceptance testing is any kind of testing that checks
that the system is acceptable to continue to the next
phase of its life cycle, usually delivery to a
customer, or making it ready for a user.

## Releasing software

* walking skeleton : bare minimum for QA to do its job
* `alpha` : version with a lot of expected bugs and new features
* `beta` : version with the new features, fewer bugs expected, bigger audience
* `release candidate` : promoting betas, checking final release can be made
* `public release`: promoting a release candidate

## Exploratory testing

This is defined as testing without a specified test plan,
where the main goal is to learn about and influence the
development of the system, as opposed to specifically
determining whether the observed behavior is equal to
the expected behavior.

Exploratory testing allows the tester to follow their own
path, thinking of edge cases on the fly or following
leads if something seems like it may cause a problem. It
also allows the tester to learn by doing, instead of
reading design documentation or code, which is often a
much more effective way to understand a system.

## A/B Testing

![](img/a-b-testing.png)


## Penetration testing

Or ... how to get a security certification

# Manual vs Automated

## Manual - pros

* It’s simple and straightforward
* It’s cheap
* It’s easy to set up
* There is no additional software to learn, purchase,
  and/or write
* It’s extremely flexible
* You are more likely to test things that users care about
  humans can catch issues that automated testing does
  not

## Manual - cons

* It’s boring
* It’s often not repeatable
* Some tasks are difficult or impossible to test manually
* Human error is a possibility
* It is extremely time- and resource-intensive
* It limits you to black-box and gray-box testing

## Automated - pros

* There is no chance of human error during test
  execution
* Test execution is extremely fast
* It is easy to execute, once the system is set up
* It is easily repeatable
* It is simple to analyze the process
* It is less resource-intensive
* It is ideal for testing some aspects of the system which
  manual testing is bad at testing
* It scales very well


## Automated - cons

* It requires extra setup time up-front

* It may not be able to catch some user-facing defects

* It requires learning how to write automated tests, as
  well as additional tools and frameworks.

* It requires more skilled staff

* Automated tests only test what they are looking for

* Tautological tests and other bad tests may creep in

