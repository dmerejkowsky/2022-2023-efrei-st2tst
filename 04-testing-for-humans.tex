---
title: Part 4 - Testing for humans
author: Dimitri Merejkowsky
date: EFREI 2022-2023
---

# Finding time to write tests

## Don't say this


> I'm going to add some tests first
> before adding this new feature.
> This means the feature will be
> ready in two weeks instead of one,
> but it will be worth it, I promise!


## The answer you'll get

> Why? It's a waste of time!
> I don't pay you to write tests, I pay you
> to add features.

## "It's a waste of time!"

It's a tricky argument to refute.

It's easier to see the time *spent*
*adding* tests, than to see the time *saved*
by *having* tests.

## My advice

Say this instead:


> The feature will be ready in two weeks

* The *way* you add the feature is up to you.
* Writing tests *is* part of the development

You would not tell a plumber which tools to use, right?

And you don't ask your boss what font to use in your IDE.

## Caveat

You're going to be less transparent about your work,
so be careful with this technique.

\vfill

The end-goal is to gain *trust*.

# Coming up with the right tests

## The three amigos

Before designing a new feature, have a meeting with:

* The stakeholder
* The dev
* The tester

Make sure you understand what the feature is
and how to test it

## BDD

A way to describe features in terms of *scenarios*

A *communication* tool more than a *development* tool.

## Gherkin

*If* you managed to get the three amigos to co-operate, and *if* you
want the stakeholder to read and write the scenarios, *then* you may
want to use
Gherkin.

But you can do BDD *without* Gherkin. The hard (and important part) - is
the communication.

## Gherkin example

```gherkin
Scenario: Making a deposit
  Given I have a bank account with a balance of 200
  When I make a deposit of 50
  Then my bank account has a balance of 250
```

## Gherkin example (2)

\small

```python
@given(
    parsers.parse(
      "a bank account with a balance of {amount:d}"
    ),
    target_fixture="account",
)
def initial_account_with_balance(
    amount,
):
    ...
@when(parsers.parse("make a deposit of {amount:d}"))
def deposit(account, amount):
    ...

@then(parsers.parse("account has a balance of {amount:d}"))
def assert_account_balance(account, amount):
    ...
```

## Gherkin example (3)

Here you get "executable documentation", which is nice.

\vfill

But again, BDD is not about the tools

# Humans and CI

## What's at stake

One a CI system is in place, you have to make sure
it stays *useful*

## Positive feedbak loop

* CI works well
* More tests are added
* More checks are done automatically
* Less regressions in the main branch

## Negative feedback loop

* CI is too long or unreliable
* Less tests are added
* CI failures start getting ignored
* More regressions in the main branch

## Failing jobs

Fix them *as soon as possible*

If the build is failing, everyone should stop working
until the build is fixed.

## Flaking tests

Either deal with them (for instance, nightly job
to find them quickly)

Or delete them.

Don't ignore the failures!

## Notifications

Make sure devs react to CI notifications.

Rotating "build farm baby sitter" might help.

## Keep tests clean

Tests should be:

* Readable
* Useful
* Rather short
* Not *too* coupled with the implementation

## Keep tests independent

* Use discipline
* Or execute them in random order

## CI

By the way, the same advice goes for:

* Compiler warnings
* Call to deprecated functions
* Time of compilations
* ...

