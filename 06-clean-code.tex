---
title: Part 6 - Clean Code
author: Dimitri Merejkowsky
date: EFREI 2022-2023
---

# Chapter one - What is Clean Code?

## Well, let's clean some code!

Example with fizzbuzz

## Start code

```java
 public static String fizzBuzzGame ( int x ) {
    String result;
    if (x % 3 == 0 && x % 5 == 0) {
        result = "fizzbuzz";
    }
    else if (x%3==0) {
        result = "fizz";
    }
  else if ( x % 5 == 0 ) {
      result = "buzz";
  }
    else {
      result = Integer.toString(x);
    }
    return result;
}
```

## End code

```java
public static String fizzBuzz(int number) {
        boolean divisibleBy3 = number % 3 == 0;
        boolean divisibleBy5 = number % 5 == 0;
        // Note: keep this test first
        if (divisibleBy3 && divisibleBy5) {
            return "fizzbuzz";
        }
        if (divisibleBy3) {
            return "fizz";
        }
        if (divisibleBy5) {
            return "buzz";
        }
        return Integer.toString(number);
    }
```

## How did we do it?

::: incremental
* Tests as a safety net
* Lots of *small* refactorings
  - Rename parameter
  - Rename method
  - Extract local variable
  - Get rid of `else if`
  - Cleanup comments
:::

See [matching PR on github](https://github.com/dmerejkowsky/fizzbuzz-refactorings/pull/1)

## A first definition


> Clean code always looks like it was written by someone who
> cares

Michael Feathers

## Why do we want clean code

> "Any fool can write code that a computer can understand.
> Good programmers write code that humans can understand."

Martin Fowler


\vfill

We want clean code because other devs will maintain it in the future.

## Code quality

One of the biggest contributors to team productivity ...

But one which is the most difficult to measure.

## Measuring code quality

Code quality is a highly *subjective* metric.

It can vary depending on the context, background and
experience of the team.

## An illustration

![](img/wtf.png)

## Sad fact

The natural tendency of any code base is for the code quality to *degrade*.

Maintaining code quality is always a challenge.

That's why you should be *continuously refactoring*

#  Advice

## Take your time!

It's usually better to have 80% well done than 100% with lower quality
and more rush.


\vfill

*Note: some teams don't apply this rule*

## Boy scout rule

Try and leave the code a little better than you found it.

## The psychopath

![](img/shining.png)


> Always code as if the person who ends up maintaining your code
> will be a violent psychopath who knows where you live.

Martin Golding

Usually, the violent psychopath is you!

## Broken Window

![](img/broken-windows.png)


## Be careful with metrics

Metrics can be gamed.

Just because it's easy to measure, does not mean it's relevant.

\vfill

On the other hand, they can be used to *discover facts* about the code
and (hopefully) *create new, good habits*.

# What the science has to say

## sleep is important

Don't work late ! And make sure to sleep well.

You'll write really bad code when you're sleep deprived, and you'll be
*unable to notice how bad you are working*.

## Code review works

Go ask other people to read and comment your code!

After all, it's how open source works :)

## Tests matter

Of course, otherwise I would not be giving this course ;)

##

... and that's all. The *science* of software development has yet to reach
consensus on *anything* but the importance of good sleep and code reviews.

And there's debate to know if there are more efficient things than code
reviews, by the way.

# Chapter two - refactoring

Demo!

# Chapter three : Clean code basics

## Convention - definition

There are several ways of doing the same thing, but the community agrees that one of the ways is better.

Note: often, arguments are used, but remember that every convention is *arbitrary*.

## Following conventions

Follow established conventions.

Python has PEP8:  https://www.python.org/dev/peps/pep-0008/

\vfill

Write code in English - unless the core domain is in French.

\vfill

Use the same tools (IDE...) as your team mates.

Careful: sometimes the conventions are *implicit*.

## Comments

Let's talk about comments.

## Comment position

Either:

- on an other line, before the described code
- on the same line, after the described code

```java
/*
example 1: - a long comment talking
about foo() and bar() ...
*/
foo();
bar();

x = spam;  // short comment about x's assignation
```

## Useless Comments (1)

Do not put useless comments:

```java
/* foo.java written by John Doe.
   Last modified on 2021-03-31
*/
```

- We already know the file name because  we've just opened it!
- We often know (or don't care) about the author name
- Ditto for date of last change (which you'll forget to update)

This is sometimes called *rotten* comments.

## Useless Comments (2)

```java
/**
* Adds 2 to to x
* @param x : the number to add 2 to.
* @return the result of adding 2 to x.
*/
int addTwo(int x) {
  return x + 2;
}
```

We can know all this by looking at the *signature of the function*!


## Dead code

Do not leave dead code:

```java
void doStuff() {
    // x = do_x_v1()
    // y = do_y_v2()
    // z = combine(x, y)
    x = do_x_v2()
    y = do_y_v2()
    z = combine_v2(x, y)
}
```

Dead code also *rots* and *smells*.

Use (or learn!) version control instead

## Useful comments

Do insert comments when the code is not enough.

Try to describe *why*, not *how*.

Further reading:

https://hackaday.com/2019/03/05/good-code-documents-itself-and-other-hilarious-jokes-you-shouldnt-tell-yourself/

## JavaDoc

```java
/**
 * In java - a special kind of comment
 * @param x
 */
void doStuff(int x) {
}
```

## When should you use documentation comments?

Lots of debate.

What I do:

- Put them in *all* public methods if you are writing a library
- Use them as high-level descriptions of tests


```java
@Test
void testComplexFeature() {
    /*
    Scenario:
    Given ..
    When ...
    Then ...
    */
    // lots of code here
}
```


##  Naming

One of the hardest problems in computer science

## Some rules

- Be consistent
- Don't use abbreviations
- Don't put the type in the name
- Use a descriptive name

```java
int age = 16  // too short
int minimal_age = 16 // nice
int minimal_age_to_be_able_to_drive  16  // to long
```

## Plural and singular

Use plural if it's a collection.

```java
for(int foo: foos) {
    ...
}
```

Or sometimes there's a nicer word:

```java
// An example from Bouygues Telecom
for(Ligne ligne: parc) {

}
```

## Use good metaphors

Use names you could explain to the rest of the team.

*Bad*: "Unlock Device"

*Good*: "Verify Identity"


## Rules for variable name size

Big scope, long name

Short scope, small name.

In general, you are allowed one-letter variables names in `for` loops
and list comprehensions but almost nowhere else.


##  Functions


## Grammar for function names

Use verbs at the imperative, present tense

```java
// Good
void displayTree() { /* ... */ }

// Bad
void displaysTheTree() { /* ... */ }

// Bad
void treeDisplay() { /* ... */ }
```

## Generic vs specific functions

The more generic the function, the shorter the name

```java
void makeCoffee(bool sugar) {
    if (sugar) {
        makeCoffeeWithSugar();
    }
    else {
        makeCoffeeWithoutSugar();
    }
}
```

## Returning booleans

Use `is`, or `has`, etc so that the code reads better:

```java
if isAllowedToDrive(person) {
   // ...
}
```


## How big should my function be?

* Functions should be really short.

## Code smell : 'and'

```java
void doBarAndFoo() {
	/* ... */
}
```

Or:

```java
void firstThingThenOtherThings() {
	/* ... */
}
```

## Code smell : "step" comments

A function that looks like this ...

```java
void myBigFunction() {
    // step 1
    ...

    // step 2
    ...

    // step3
    ...
}
```

... can probably be split in 3 (the comments are the clue)

## Code smell :  Number of arguments

- 0 to 3 : probably fine
- more than 3 : danger -> consider introducing a struct or class for storing the parameters


## Example - too many parameters


```java
public class Coffee {
    public void orderCoffee(
        boolean tall, boolean milk, boolean noSugar)
    {
        makeCoffee(tall, milk, noSugar);
        serve();
    }

    private void makeCoffee(
        boolean tall, boolean milk, boolean noSugar)
    {
        if (tall)  addWater();
        if (milk) addMilk();
        if (!noSugar)  addSugar();
    }
}
```

## Example - using a record for the command

```java
// New ↓
record Order(boolean tall, boolean milk, boolean sugar) {
}

public class Coffee {
   public void orderCoffee(Oder older) {
        makeCoffee(order);
        serve();
    }

    // Note <- only one parameter now
    private void makeCoffee(Order order) {
        if order.tall() addWater();
        if order.milk() addMilk();
        if order.sugar() addSugar();
    }
}
```

## Avoid double negative

Notice how we went from

```java
if(!noSugar)
```

to

```java
if(order.sugar()) {
}
```

which is more readable.

Better to have a variable with default value of `true`,
rather than a negative variable with a default value of `false`.

## More or less lines?

::: incremental
* It depends!
:::

## Less lines is better

```java
if (canVote) {
    return true;
}
else {
    return false;
}
```

vs

```java
return canVote;
```

## More lines is better

```java
if (
    age >= 18
    && nationality == "French"
    && electorsList(sector).contains(name);
) {
   System.out.println("can vote");
}
```

vs

```java
bool adult = age >= 18;
bool french = nationality == "French";
bool registered = electorsList(sector).contains(name);

if (adult && french && registered) {
   System.out.println("can vote");
}
```

## Early return - Before

\small
```java
String returnStuff(SomeArgument arg1, SomeArgument arg2) {
  if (arg1.isValid()) {
    if (arg2.isValid()) {
      SomeObject object = doStuff(arg1, arg2);
      if (object != null) {
        return "Stuff";
      } else {
        throw new RuntimeException(); // object was null
      }
    } else {
      throw new RuntimeException(); // arg2 was not valid
    }
  } else {
    throw new RuntimeException(); // arg1 was not valid
  }
}
```

::: incremental
* "Happy path" in the middle and to the right
* Cause of the exception far from the throw
:::

## Early return - After

\small
```java
String returnStuff(SomeArgument arg1, SomeArgument arg2) {
  if (!arg1.isValid()) {
    throw new RuntimeException();
  }
  if (!arg2.isValid()) {
    throw new RuntimeException();
  }

  SomeObject object = doStuff(arg1, arg2);

  if (object == null) {
    throw new RuntimeException();
  }

  return "stuff";
}
```

::: incremental
* Comments are gone
* Less horizontal space taken
* "Happy path" *after* the "sad path"
* Cause of exception right before the throw
:::

# Chapter four : Clean code and Classes

## Foreword

Classes, composition and inheritance are powerful but *dangerous* tools.

* They hide the control flow.
* It's very easy to make a mess!

## Naming

Classes names are *nouns* and start with an uppercase letter.

Avoid meaningless words like `Handler`, `Manager`, `Data`, `Info` ...

## Use inheritance with caution

Lots of powerful and dangerous features:

* Multiple inheritance (for languages that have them)
* Some attribute may be shared (ditto)

##  Advice

- Only every inherit from *one* parent
- Use inheritance for exceptions
- Use inheritance for abstract base classes (more on this later)

And that's all!

# Chapter five : Packages and modules

## Naming

Use short names, in `lower_case`

## Avoid stutter

```java
package corp.acme.fraction;
public class Fraction {
  // ...
}
```

```java
// Bad:
static Fraction addFractions(Fraction fractionA, Fraction fractionB) {
    // ...
}
```

```java
// Better
static Fraction add(Fraction a, Fraction b) {
    // ...
}
```

# Chapter six: Clean architecture

## What is architecture?

the collection of modules, classes and methods that will
be used throughout the project.

This is more high-level than the *contents* of the classes, modules, and functions.

## Rules of architecture

The _rules_ of architecture do not depend on the language.

Software is made of:

- sequence
- selection
- iteration

(It never changed for the last 50 years)

So if the nature of the code did not change, the rules of architecture did
not change either.

## Architecture

Architecture, design, and programming are the same thing - the only difference
is how high-level they are.

Corollary: architects who do not write code, or programmers who do not think about
high-level design are not doing their best.

##  APIs

## Definition

Application Programming Interface

A set of functions (or methods) or class you can use
as an *external* user of a piece of code

## APIs and libraries

Often you use an API from a library.

Example with `datetime`:

https://docs.python.org/3/library/datetime.html


## What makes good APIs

* Good naming
* Good metaphors
* Consistency
* Easy to use
* Hard to misuse
* Simple things should be easy, complex things should be possible


## Making good APIs

It's *hard*

What can help:

* Brainstorms
* Tests
* Documentation
* Examples
* Review

And you should do all  of this *before* writing the production code, because
architecture is much harder to change afterwards!

## The big problem with APIs

They are hard to change.

Sometimes they *break*, which means *all the code that use them* must change,
and that can have really bad consequences.

## The bad news

As soon as you *write a function* in a piece of code, you *are* defining
an API from the point of view of the rest of the code.

All of the above applies (minus the fact breaking them is not as bad)

## Advice

Treat any piece of code as it was part of a public library usable by anyone.

## Object Oriented Programming

One definition: put *data* and *behavior* that operate on this data in
the same place.

Note: not necessarily achieved with classes. It's mostly a way
of *structuring* code

## Encapsulation

The big idea: users of a class should not *need to know* about
how it works internally.

## Example: the counter class

```java
public class Counter {
    private int _tally;

    public Counter() {
        _tally = 0;
    }

    public void add(int count) {
        _tally += count;
    }

    public int getTotal() {
        return self._tally;
    }
}
```

- `Counter(), add(), total()`: public interface
- `_tally`: private implementation detail


## An other example

```java
public class Person {
    String firstName;
    String lastName;

    public Person(String firstName, String lastName) {
        this.firstName = firstName;
        this.lastName = lastName;
    }

    public String fullName() {
        return firstName + " " + lastName;
    }
}
```
From outside the `Person` class:

* `firstName` and `lastName` are read/write
* `fullName` is read-only

## Invariant

We say the class enforces an *invariant*

The full name will *always* be the concatenation of `firstName`
and `lastName`.

The tally will *always* contain the running total of calls to `add`.

## Basic principles

1. Passes all the tests
1. Minimize duplication
1. Maximize clarity
1. Fewer elements


## How to get clean architecture

* Make it work, make it right, make it fast
* Use TDD
* Always look back at the architecture

## More advanced principles

* Single Level of Abstraction
* Law of Demeter
* Tell, don't ask - Command/Query Separation
* Calisthenic objects
* SOLID
* DDD tactics

...

And more

## Advice for principles

* They are abstract
* They often contradict each other

Don't try and follow all of them at once!

## Law of Demeter

Formulation:

You should *not* access the *fields of the field* of an other class

## Example - Employee

```java
record Department(String name) {}

class Employee {
    private String name;
    private String department;

    public Employee(String name) {
        this.name = name;
        this.department = null;
    }

    public Employee(String name, Department department) {
        this.name = name;
        this.department = department;
    }
}
```

## Example - EmployeeDirectory

```java
class EmployeeDirectory {
  public static String describeEmployee(Employee employee) {
    String employeeName = employee.name;  // OK;
    String departmentName;
    if(employee.department != null) {
       departmentName= employee.department.name ; // Bad!
    }
    if(departmentName != null) {
      return employeeName + " from  " + departmentName;
    } else {
      return employeeName;
  }
}
```

## Why is the law of Demeter a good idea?

We have a coupling between `describeEmployee()` and `Employee`  (which is fine)

But the `describeEmployee()` method *knows* that:

- employees belong to 0 or 1 department
- every department has a name

(which is tighter coupling and *not* fine)

## When the specifications change

Let's say employee can belong to *one or more* departments.

```diff
public class Employee {
- private Department department;
+ private ArrayList<Department> departments;
}
```

Now we need to change `describeEmployee`, which a sign of *fragility*.

## When the specifications change

```diff
String describeEmployee(Employee employee) {
    String employeeName = employee.name;
    String departmentName;
-    if(employee.department != null) {
-       departmentName= this.department.name;
-    }
+    if(!employee.departments.isEmpty()) {
+        Department department = employee.departments[0];
+        departmentName = department.name;
+    }
}
```

## The fix

Move the coupling from `describeEmployee` and `Department` inside `Employee`

```java
class Employee {
    // New!
    String getDepartmentName() {
        if (this.department != null) {
            return department.name;
        }
        return null;
    }
}
```

## The fix (2)

```java
// No more Demeter violation here
static String describeEmployee(Employee employee) {
    String employeeName = employee.name;
    String departmentName = employee.getDepartmentName();
    if (departmentName != null) {
        return employeeName + " from " + departmentName;
    } else {
        return employeeName;
    }
}
```

## Same change, different code

```diff
class Employee:
   String getDepartmentName() {
-       if (this.department != null) {
-           return this.department.name;
-       }
+       if(!this.departments.isEmpty()) {
+           firstDepartment = this.departments[0];
+           return firstDepartment.name;
+      }
    }
}
```

::: incremental
* `describeEmployee()` in the other class - `EmployeeDirectory` - did not need to change!
:::


## Before

![](img/demeter-1.png)

## After

![](img/demeter-2.png)



## Command/Query separation

- Command: change state
- Query : return some state

## Command/Query separation

Either:

Command: returns nothing, has side effects

Query : returns something, has no side effects

Avoid side-effects in a function that returns a value!

## SRP

Single responsibility principle

A class/module should have only *one* reason to change.

- What are the things that can change?
- Who might be asking for these changes?


## SRP

Smells:

- A place where code keeps changing
- Conflicts happening in the same file
- Rigid code

##  Going further

- SOLID
- CUPID: https://dannorth.net/2022/02/10/cupid-for-joyful-coding/
- Primitive Obsession
- Behavior Drive Development
- Domain Driven Design

