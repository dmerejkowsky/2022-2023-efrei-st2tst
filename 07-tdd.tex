---
title: Part 7 - Test Driven Development
author: Dimitri Merejkowsky
date: 2022-2023 @ EFREI
---

#  Introduction

## My time at Aldebaran

The qiBuild project

* 6 years
* ~= 40k lines of Python code
* 3,286 commits
* 158,703 lines added
* 120,441 lines removed

## The qiBuild project

Could not afford bugs

Could not afford long delays before writing new features

End-users right next to me ...

## Before TDD

I hate doing manual testing, so I wrote lots of tests

But they were:
* Slow
* Hard to maintain
* Not very good at catching bugs

And changing architecture looked scary because it would mean rewriting all the tests

## After TDD

Tests were:

* Fast
* Good at catching bugs
* Helped me achieve a nice architecture

I still did beta-test releases of qibuild, but very few bugs managed to
slip through – and they were promptly fixed anyway.

It was easy to add new features, even for new contributors

# TDD

## What is TDD?

* A technique to make test and production code evolve together
* Something that changed the way I write software for ever
* Not for everyone or for every situation
* Worth trying

And don't feel bad if you don't use it!


## Feedback - 1

![](img/feedback-1.png)

## Feedback - 2

![](img/feedback-2.png)


## The cycle

![](img/tdd.png)

## The rules

* You shall not write production code *except* to make a failing test pass
* You shall not write more production code than is required to make a failing test pass
* You shall not write more test code than is required to provoke a failure (any kind of failure)
* Once all test pass, you *have* to stop and consider a refactoring


# Tips and Tricks

## Keep trying

Keep your cycles shorts

Use ZOMBIES (zero/one/many/boundary/interface/error/scenario) to know
which test to write next

Learn how to not get stuck

In takes practice to master TDD, but don't expect results
before quite some time!


## On test frameworks

Use a framework made for your language

* Avoid Rspec-like frameworks unless you are using Ruby
* Avoid JUnit-like frameworks unless you are using Java
* Use pytest when using Python

In doubt, use the *simple* one.

Don't be afraid to write your own test "helpers".

## Isolation

Each test can be run independently of the others.

They should be ran in parallel

Ideal goal: if there's one bug in the production code, exactly one
test fails and tells you where to look.

## Structure

* Arrange
* Act
* Assert
* Annihilate (optional)

## One assertion per test

More than a *logical* assertion - you are allowed to use
the ``assert`` keyword more that once

Avoid `act,assert,act,assert`

## Write the scenario down

```java
/**
  Scenario:
  - Create a file named a.txt in the local folder with
    "old" contents
  - Add a new version of the `a.txt` file in the remote folder
  - Synchronize the remote and local folders
  - Check that `a.txt` has been updated
*/
@Test
public void can_update_local_file() {
    // ...
}
```

## Baby steps

![](img/baby-steps.png)

## Many Much More Smaller Steps

![](img/mmmss.excalidraw.png)

=> https://www.geepawhill.org/2021/09/29/many-more-much-smaller-steps-first-sketch/

# Demo - Fractions

$f_1 = \frac{n_1}{d_1}$

$f_2 = \frac{n_2}{d_2}$

$f_1 + f_2 = \frac{n_1.d_2 + n_2.d_1}{d_1.d_2}$

## Software writer and accountants

In both cases, a tiny error can make the company loose millions

Accountants are certified and must use the "double entry book-keeping" method.

\vfill

Why not the same standard for software devs?

## Double Loop

![](img/tdd-double-loop.png)

## TDD Killer features - 1

* Easy to organize pair programming in ping pong mode
* Emerging design and solution
* All business code is tested
* I can stop and restart my work at any time
* TDD is fun
* Release is cool
* No dead code
* Better interest for business

## TDD Killer features - 2

* Coverage is no more a concern
* Bugs easier to handle
* Bug detected earlier
* Better predictability
* No more debugger to use
* Code API is cleaner
* Fast feedback
* Documentation
* Better confidence in code and tests
* Easy to know when to stop coding

## Starting TDD

Don't ask permission!

There are meetups you can attend - ask me
for the details
