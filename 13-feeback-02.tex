---
title: Part 13 - Second workshop feedback
author: Dimitri Merejkowsky
date: EFREI 2022-2023
---

# Ohce Feedback

Let's see various solutions

##

```diff
-  greet () {
-    const currentHour = this.clock.currentHour()
-    if (currentHour >= 6 ...) {
-    // ..
-  }
+  greet (currentHour)
+    if (currentHour >= 6 ...) {
+    // ..
+  }
```

Split pure functions and IO

This is a central idea in languages like Haskell :)

## Same idea

`mainLoop()` does all the I/O

```javascript
class Interactor
  mainLoop() {
   while(true) {
     const input = this.readInput();
     const { output, keepGoing } = \
       this.handleInputLogic(input);
     if (!keepGoing) {
        return False
     }
     this.printMessage(output)
  }
}
```
##

`handleInputLogic()` is all computation, no side effects

```javascript
handleInputLogic(input) {
  if (input === 'quit') {
    return { keepGoing: false }
  }
  const reversed = reverse(input)
  if (input === reversed) {
    return {
      keepGoing: true,
      output: reversed + " That was a palindrome"
    }
  }
  else {
    return {
      keepGoing: true,
      output: reversed
    }
  }
}
```

## Monkey patch

```python
def returns_0( ):
    return 0

def test_nightly_greeting():
    """
    Assert that greeter says "Good night" at midnight
    (when current hour is 0)
    """
    greeter = Greeter()
    greeter.clock.current_hour = returns_0
    assert g.greet() == "Good night"
```

\vfill

Technique worth to know about, but use it in last resort.

## Monkey patch with jest

```javascript
const greeter = new Greeter()
jest
  .spyOn(greeter.clock, 'currentHour')
  .mockReturnValue(0)
expect(greeter.greet()).toBe('Good night')
```

\vfill

A little bit better :)

##

Some of you redirected stdin and stout to test the UI.

This is also a neat trick :)

## Nope

```python
class SystemClock:
    def current_hour(self):
        now = datetime.now()
        return now.hour
    def current_hour(self, hour=None):
        if hour != None :
            now = hour
        else:
            now = datetime.now().hour
        return now
```

You have production code and test code in the same class!


## Stil nope


```python
class SystemClock:
    def __init__(self, current_hour=None):
        self._hour = current_hour

    def current_hour(self):
        if self._hour is not None:
            return self._hour
        else:
            return datetime.now().hour
```

Building a `SystemClock` with a pre-defined hour
makes no sense.

## Almost there ...

```cs
public interface GiveHour
{
  public int GetCurrentHour();
}

public class Return_9_Oclock : GiveHour {
  public int GetCurrentHour()
  {
     return 9;
  }
}
```

##

```cs
Greeter greet = new Greeter(new Return_9_Oclock());
Assert.That(greet.Greet(), Is.EqualTo("Good morning"));
```


## A brand new test class

```javascript
const FakeSystemClock {
  constructor(hour) {
    this.hour = hour
  }

  currentHour() {
    return this.hour
  }
}
```

Now we have a class for all our testing needs

## And then..

Just a tiny change in the constructor:

```diff
class Greeter {
-  constructor() {
-     this.clock = new SystemClock()
-  }
+  constructor (clock = new SystemClock()) {
+    this.clock = clock
+ }
```

\vfill

```javascript
test('midnight', (t) => {
  const fakeClock = new FakeSystemClock(0);
  const greeter = new Greeter(clock=fakeClock})

  t.equal(greeter.greet(), "Good night")
})
```
