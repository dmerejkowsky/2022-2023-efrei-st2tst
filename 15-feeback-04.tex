---
title: Part 15 - Project feedback (1)
author: Dimitri Merejkowsky
date: EFREI 2022-2023
---

# General feedback

## Reminder

Please make sure your team letter appears in the github URL !

Do it now so that you don't forget - don't worry, I'll still have access
and the old URL will continue to work.

## Reminder (2)

Same thing for the filenames! I got at least 5 "test plan.xlsx"

## On deliveries

One file per delivery, please.

Make sure the log is separated from the rest from now on
(most of you already did that)

# Test plans


## Some good ideas

Have the test runs as different versions of an md file under
version control :)

Include screenshots

Include videos

Add links from the test plan to the bug tracker

## Careful with images and video

Text takes very little space and is easy to search and index.

\vfill

This is not the case for images and videos, which take more space
and are less accessible in general.

## Tests plans in excel

What do *you* think?

## Tests plans in excel

It works but it's not very pleasant.

Many mistakes possible - in real life, manual testers often use a
dedicated tool.

Need to strike a balance between too much or not enough details for
manual tests.

# Bug tracker

## Some good ideas

* Include URLs of the pages that present the bug

* Using question marks for "not sure this is a bug" situations

## Titles

## Titles (1)

Don't do that:

```text
Bug #15 - 14 - cannot demote employee
```

\vfill

Let the bug tracker generate issue numbers!

## Titles (2)

What's wrong with this title?

\vfill

> When creating a new team with empty name

## Titles (3)

What's wrong with this title?

\vfill

> Second line doesn't update

## Titles (4)

We want both the condition and the nature of the bug

Lots of ways to phrase it:

\vfill

> Prohibit adding to no team

\vfill

> Crash when adding to non-existing team

\vfill

> Team creation: add a team without value shows traceback


## Titles (5)

![](img/redundant-bug-label.png)

:::incremental
* What's wrong here?
* Redundant bug label
:::


## Issues description

A pattern often used:

```text
Step to reproduce:
* Go to the 'teams' page
* Click on the "delete" button next to a team containing
  members

Expected:
All the members of the team now belong to no team

Actual:
All the members of the team get deleted as well
```

\vfill

Not mandatory, though.

## Bug lifecycle (1)

```text
Adresse 2 identique à la 1ère (#4)

* Comment 1
  2ème ligne identique à la 1ère

* Comment 2
  Quand on modifie l'adresse 2, il prend la valeur
  de l'adresse 1 mais si on modifie l'adresse 1,
  l'adresse 2 n'est pas modifiée
```

:::incremental
* What's wrong with those comments?
* Does not look like they were written by someone who cares
:::

## Bug lifecycle (2)

```text
Delete a team with members (#5)
-------------------------------

* Comment 1
  The deleting of the team which contains members doesn't work
  hen we click on the delete button, an 500 error is returned by the
  server and the team is not deleted.

* Comment 2
  The team is now deleted, but the employees of the team are also
  deleted from the database
```

:::incremental
* What's wrong with those comments?
* Missing information version
:::

## Bug lifecycle (3)

When you close a bug, always indicate the version for which it was fixed!

Don't delete closed bugs

## Some advice

Opinion: the simpler the better - looking at you, JIRA.

For instance:

* don't use labels unless you have to
    * for instance, one label per version is probably overkill
* don't configure state machines for the workflows

Rationale: if the bug track is too painful to use, it won't be used
and in the end, useless.
